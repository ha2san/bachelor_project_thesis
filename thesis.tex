%%%%%%%%%%%%%%%%%%%%%%%%%%%%%%%%%%%%%%%%%%%%%%%%%%%%%%%%%%%
% EPFL report package, main thesis file
% Goal: provide formatting for theses and project reports
% Author: Mathias Payer <mathias.payer@epfl.ch>
%
% This work may be distributed and/or modified under the
% conditions of the LaTeX Project Public License, either version 1.3
% of this license or (at your option) any later version.
% The latest version of this license is in
%   http://www.latex-project.org/lppl.txt
%
%%%%%%%%%%%%%%%%%%%%%%%%%%%%%%%%%%%%%%%%%%%%%%%%%%%%%%%%%%%
\documentclass[a4paper,11pt,oneside]{report}
% Options: MScThesis, BScThesis, MScProject, BScProject
\usepackage[BScThesis,lablogo]{EPFLreport}
\usepackage{xspace}

\title{Retrofitting defences to C++ code}
\author{Hassan Habib}
\supervisor{Nicolas Badoux}
\adviser{Prof. Dr. sc. ETH Mathias Payer}
%\coadviser{Second Adviser}
%\expert{The External Reviewer}

\newcommand{\sysname}{Retrowrite\xspace}

\begin{document}
\maketitle
%\makededication
%\makeacks

\begin{abstract}
The \sysname tool enables lateral decomposition of a multi-dimensional
flux compensator along the timing and space axes.

The abstract serves as an executive summary of your project.
Your abstract should cover at least the following topics, 1-2 sentences for
each: what area you are in, the problem you focus on, why existing work is
insufficient, what the high-level intuition of your work is, maybe a neat
design or implementation decision, and key results of your evaluation.

\end{abstract}

%\begin{frenchabstract}
%For a doctoral thesis, you have to provide a French translation of the
%English abstract. For other projects this is optional.
%\end{frenchabstract}

\maketoc

%%%%%%%%%%%%%%%%%%%%%%
\chapter{Introduction}
%%%%%%%%%%%%%%%%%%%%%%

The introduction is a longer writeup that gently eases the reader into your
thesis~\cite{dinesh20oakland}. Use the first paragraph to discuss the setting.
In the second paragraph you can introduce the main challenge that you see.
The third paragraph lists why related work is insufficient.
The fourth and fifth paragraphs discuss your approach and why it is needed.
The sixth paragraph will introduce your thesis statement. Think how you can
distill the essence of your thesis into a single sentence.
The seventh paragraph will highlight some of your results
The eights paragraph discusses your core contribution.

This section is usually 3-5 pages.

%%%%%%%%%%%%%%%%%%%%
\chapter{Background}
%%%%%%%%%%%%%%%%%%%%
In this section, we will give a little bit of backround in order to understand
the experience

\section{Compiler}
Computer programs can be executed in two different ways: they can be interpreted
by a software (an interpreter) or they can be compiled into machine code that can
be then be executed by the hardware.

In the second approach, a compiler is needed to translate the source code into binary code.
Let's have a quick overview of what is contained in a binary executable.

\section{Binary executable}
\subsection{ELF format}
%source man page elf
The dominant executable format for binary in Unix system is the ELF\footnote{Executable
and Linkable Format} format. 

An executable file respecting the ELF format must follow a certain layout. At
first it must have an ELF header followed by a program header table or a section
header table, or both. 

There exist multiple tools that can be used to display information about an ELF file. 

\section{Binary rewritting}
Binary rewritting consists of modifying without the source code an executable
in a way that the program still behave as expected. It can be used for multiple
reasons as for example for program optimization, program obfuscation or as in
the case of Retriwrite for security policy enforcement. 
%show example of obfuscation

Binary rewritter can be divided into two main approach: Dynamic instrumetation
and static instrumentation

\subsection{Dynamic instrumentation}
In this scheme, the rewritting is done while the program is executed. In order
to analyse the binary during the execution, it uses the PTRACE API in Linux for
example.  Only the parts executed can be analysed and rewritten.
\subsection{Static instrumentation}
Here, the rewritter operates on binaries stored in persistent memory
(executables that are not being run). In outputs a new binary executable that
corresponds to the initial executable but rewritten. these executables have the
advantage of being almost as fast as the original executable that would have
been compiled for instrumentation.
\subsection{Four steps}
There is mainly four steps for binary rewritting:
\begin{enumerate}
    \item Parsing
        %elaborate
    \item Analysis
        %elaborate
    \item Transformation
        %elaborate
    \item Code generation
        %elaborate
\end{enumerate}
\subsection{Techniques}
There are several techniques for the transformation parts. 





\section{Retrowrite}
Speak about retrowrite. What it can do and what it cannot do (maybe in challenges)



%%%%%%%%%%%%%%%%
\chapter{Challenges=Goal}
%%%%%%%%%%%%%%%%
Speaking about state of the art rewritter, and mostly
speak about retrowrite

I will explain the main issue with C++

%%%%%%%%%%%%%%%%%%%%%%%%
\chapter{Implementation}
%%%%%%%%%%%%%%%%%%%%%%%%
Where I speak about the testing part. How I got the c++ packages, the issue with them
I can first explain the experiment where a test a bunch of debian packages and
then I can take specific program and test them manually

I can choose specific program (bitcoind) and show the exact problems

I can also on simple program written by me where it
doesn't work (exception)



%%%%%%%%%%%%%%%%%%%
\chapter{Evaluation}
%%%%%%%%%%%%%%%%%%%
I can show graph about debian packages, and also explain the main issue with the big program



%%%%%%%%%%%%%%%%%%%%%%%
%\chapter{Related Work}
%%%%%%%%%%%%%%%%%%%%%%%
%
%I can cite other rewritter.



%%%%%%%%%%%%%%%%%%%%
\chapter{Conclusion}
%%%%%%%%%%%%%%%%%%%%

In the conclusion you repeat the main result and finalize the discussion of
your project. Mention the core results and why as well as how your system
advances the status quo.

\cleardoublepage
\phantomsection
\addcontentsline{toc}{chapter}{Bibliography}
\printbibliography

% Appendices are optional
% \appendix
% %%%%%%%%%%%%%%%%%%%%%%%%%%%%%%%%%%%%%%
% \chapter{How to make a transmogrifier}
% %%%%%%%%%%%%%%%%%%%%%%%%%%%%%%%%%%%%%%
%
% In case you ever need an (optional) appendix.
%
% You need the following items:
% \begin{itemize}
% \item A box
% \item Crayons
% \item A self-aware 5-year old
% \end{itemize}

\end{document}
